\documentclass{article}
\usepackage[utf8]{inputenc}
\usepackage{geometry}
\usepackage{graphicx}
\geometry{a4paper, margin=1in}

\title{Dokumentace k semestrální práci KIV/UPS Server-Klient Blackjack}
\author{Jan Vandlíček}
\date{\today}

\begin{document}

\maketitle

\section{Základní popis hry blackjack}
Projekt implementuje klasickou hru Blackjack, ve které hráči snaží dosáhnout součtu hodnot karet blížícího se 21, aniž by překročili tuto hodnotu. Server spravuje hru a komunikuje s klienty, kteří představují hráče. Hra je implementována bez dealera a eso má hodnotu "1".

\section{Popis protokolu}
\subsection{Formát zpráv}
Zprávy jsou formátovány jako textové řetězce s předem definovanými příkazy a parametry. Každá zpráva začíná heslem, následovaným příkazem a daty specifickými pro daný příkaz.

Obecný formát: \texttt{<heslo><delka\_parametru><prikaz><parametr>}. První zpráva, kterou je nutno poslat z klienta: \texttt{420BJ69005NICKVandl} - heslo: 420BJ69; počet znaků parametru příkazu: 005; příkaz: NICK, parametr: Vandl.

\subsection{Možné zprávy}

Příklad hierarchie zpráv klienta:
\begin{itemize}
    \item \texttt{420BJ69005NICKVandl}
    \item \texttt{420BJ69006JOINGame 1}
    \item \texttt{420BJ69000PLAY}
    \item \texttt{420BJ09003TURNHIT}
    \item \texttt{420BJ09005TURNSTAND}
\end{itemize}

Všechny použité příkazy pro klient a server:
\begin{itemize}
    \item \textbf{NICK}: Příkaz pro nastavení přezdívky (nickname).
    \item \textbf{PING}: Příkaz pro odeslání ping zprávy.
    \item \textbf{PONG}: Příkaz pro odpověď na ping zprávu.
    \item \textbf{JOIN}: Příkaz pro připojení k serveru/hře.
    \item \textbf{PLAY}: Příkaz pro spuštění hry.
    \item \textbf{GMIF}: Příkaz pro získání informací o hrách.
    \item \textbf{GMJN}: Příkaz pro připojení k hře.
    \item \textbf{GMCK}: Příkaz pro ověření, zda může začít hra.
    \item \textbf{GMST}: Příkaz pro spuštění hry.
    \item \textbf{GMEN}: Příkaz pro ukončení hry.
    \item \textbf{TURN}: Příkaz pro provedení tahu v hře.
    \item \textbf{NEXT}: Příkaz pro zahájení dalšího kola hry.
    \item \textbf{STOP}: Příkaz pro zastavení akce nebo procesu.
    \item \textbf{RETR}: Příkaz pro získání stavu hry.
    \item \textbf{STAT}: Příkaz pro nastavení stavu.
    \item \textbf{KILL}: Příkaz pro ukončení nebo zastavení.
    \item \textbf{KIL2}: Další variace příkazu pro ukončení nebo zastavení.
\end{itemize}

\subsection{Přenášené struktury a datové typy}
Struktury zahrnují \texttt{Player}, \texttt{Game}, \texttt{Hand}, \texttt{Card} a další. Datové typy zahrnují celá čísla, řetězce a seznamy.

\subsection{Význam a kódy přenášených dat}
Každý datový prvek má specifický význam, například \texttt{Player} obsahuje informace o hráči, \texttt{Game} o stavech hry.

\subsection{Omezení a validace dat}
Data jsou validována pro integritu a platnost, například hodnoty karet nesmí překročit určité meze.

\subsection{Stavy a okna hry}

Grafické rozhraní zobrazuje hru pomocí tří obrazovek, login screen, game list screen a gameplay screen. První obrazovka, login screen, slouží k zadání IP adresy a portu serveru a přezdívky hráče. Obrazovka game list screen se zobrazí jako druhá, po úspěšném připojení klienta na server, klient v tomto okně vidí všechny dostupná lobby a může se do nich dvojklikem připojit. Gameplay screen je poslední obrazovka, ve které probíhá hra jako taková.

\begin{enumerate}
    \item Login screen
	\begin{itemize}
	    \item \texttt{<420BJ69><delka\_parametru><NICK><parametr>}
    	\end{itemize}
    \item Game list screen
	\begin{itemize}
	    \item \texttt{<420BJ69><delka\_parametru><JOIN><parametr>}
	    \item \texttt{<420BJ69><000><PLAY>}
	\end{itemize}
    \item Gameplay screen
	\begin{itemize}
	    \item \texttt{<420BJ69><003><TURN><HIT>}
	    \item \texttt{<420BJ69><005><TURN><STAND>}
	\end{itemize}
\end{enumerate}

Diagram zobrazuje postupnou výměnu zpráv mezi klientem a serverem pro různé fáze hry.

Od doby, co se klient úspěšně připojí klient na server, je mu posílána (ze strany serveru) zprává "PING", na kterou klient automaticky odpovídá zprávou "PONG", výměna těchto dvou zpráv zjišťuje stav připojení klienta a serveru.

Pokud se klient korektně přihlásí na server pomocí IP adresy, portu a nastaveného nicku, přejde z login screen do game list screen, je mu zaslána zpráva \texttt{<420BJ69><delka} \texttt{\_parametru><GMIF><parametr>}, parametrem této zprávy jsou herní místnosti, jejich obsazenost a maximální kapacita. Pokud hráč úspěšně vstoupí do místnoti, je mu zaslána zpráva \texttt{<420BJ69><001><GMIF><1>} společně se zprávou \texttt{<420BJ69><005><GMCK><indikator\_zda\_je\_možno\_hru\_spustit|aktualni\_obsazenost>|} \\ \texttt{<maximalni\_kapacita>}. Když jsou v lobby aspoň dva hráči, je možno hru spustit.

Po spuštění hry přijde hráčům zpráva obsahující jejich karty a hodnotu a nicky ostatních hráčů, zpráva má tento tvar \texttt{<420BJ69><delka\_parametru><GMST><nicky|karty|celkova\_hodnota\_karet}.

Při každé zprávě PING--PONG je hráčům v průběhu hry zaslán stav připojení všech hráčů ve tvaru \texttt{<420BJ69><delka\_parametru><STAT><nick|stav>}.

Po provedení tahu odesílá hráč zprávu \texttt{<420BJ69><003><TURN><HIT>} nebo \texttt{<420BJ69><005><TURN><STAND>}, podle toho jaký tah zvolil. Pokud odehrajou všichni hráči svůj tah, server pošle zprávu o aktualizovaném stavu karet apod. Zpráva má tento tvar \texttt{<420BJ69><delka\_parametru><NEXT><nicky|aktualizovane}
\texttt{\_karty|celkova\_hodnota\_karet>}.

Pokud všichni hráči složí karty (zahrají tah "STAND"), tak hra končí a klientům je zaslána zpráva o konci hry a s jejím výhercem \texttt{<420BJ69><delka\_parametru><GMEN><nick\_vyherce>}.

Po skončení hry, jsou hráči vráceni na druhou obrazovku game list screen a je jim opět zaslána zpráva o stavech herních místností.

\begin{figure}[h] % Specifikuje umístění obrázku (h - here, t - top, b - bottom, p - separate page)
    \centering % Zarovná obrázek na střed
    \includegraphics[width=0.6\textwidth]{img/diagram.png} % Specifikuje název a relativní šířku obrázku
    \caption{Stavový diagram pro hru Blakcjack} % Popisek nebo titulek obrázku
    \label{fig:obrazek} % Identifikátor pro odkazování na obrázek
\end{figure}

\section{Popis implementace klienta a serveru}
\subsection{Dekompozice do Modulů/Tříd}
Server i klient jsou rozděleni do několika modulů, každý s vlastní funkcionalitou.

\subsubsection{Server (Go)}
\begin{itemize}
  \item \textbf{server.go}: Hlavní modul serveru, zodpovědný za inicializaci hry, správu herních místností a zpracování připojení klientů.
  \item \textbf{game.go, player.go, hand.go, deck.go, card.go}: Tyto moduly definují základní struktury hry, včetně hráčů, karet, balíčků a herních rukou.
  \item \textbf{table\_status.go}: Obsahuje strukturu pro udržení stavu herního stolu.
  \item \textbf{comm.go, communication.go}: Zabývají se komunikací a zasíláním zpráv mezi serverem a klienty.
\end{itemize}

\subsubsection{Klient (Python)}
\begin{itemize}
  \item \textbf{main.py}: Hlavní spouštěcí bod aplikace klienta, zahajuje uživatelské rozhraní a zpracovává události.
  \item \textbf{client\_to\_server\_message.py}: Obsahuje funkce pro tvorbu a zpracování zpráv odesílaných serveru.
  \item \textbf{validation.py}: Zajišťuje validaci vstupních dat a zobrazuje upozornění.
  \item \textbf{msg\_const.py}: Definuje konstanty a parametry pro zprávy používané v komunikaci s serverem.
\end{itemize}

\subsection{Rozvrstvení aplikace}
Server a klient jsou strukturováni do vrstev, zahrnujících síťovou komunikaci, logiku hry a uživatelské rozhraní.

\subsubsection{Server (Go)}
\begin{itemize}
  \item \textbf{Síťová Vrstva}: Zpracovává síťové požadavky a komunikuje s klienty.
  \item \textbf{Logická Vrstva}: Obsahuje herní logiku a pravidla Blackjacku.
  \item \textbf{Datová Vrstva}: Ukládá a spravuje stav hry, hráče a herní objekty.
\end{itemize}

\subsubsection{Klient (Python)}
\begin{itemize}
  \item \textbf{Grafické Uživatelské Rozhraní (GUI)}: Umožňuje hráči interakci s hrou prostřednictvím grafického rozhraní.
  \item \textbf{Komunikační Modul}: Zajišťuje komunikaci s herním serverem.
  \item \textbf{Logická Vrstva}: Zpracovává herní logiku na straně klienta, jako jsou herní rozhodnutí a akce hráče.
\end{itemize}

\subsection{Použité knihovny a verze prostředí}
Využívá se GO 1.21 pro server a Python 3.9.2 spolu s knihvnou Tkinter pro klienta.

\subsection{Metoda paralelizace}
Server využívá gorutiny pro paralelní zpracování požadavků klientů.

\subsubsection{Server (Go)}
\begin{itemize}
  \item Využívá \textbf{gorutiny} pro asynchronní a paralelní zpracování. Například, server může současně spravovat více herních místností a klientů, každý běžící v samostatné gorutině.
  \item \textbf{Synchronizace}: Pro zajištění bezpečnosti při přístupu k sdíleným zdrojům používá mutexy a další synchronizační mechanismy.
\end{itemize}

\subsubsection{Klient (Python)}
\begin{itemize}
  \item Ačkoliv konkrétní metody paralelizace nejsou v poskytnutých souborech explicitně uvedeny, v GUI aplikacích běžně dochází k asynchronnímu zpracování událostí. 
  \item Může využívat vlákna nebo asynchronní funkce pro zpracování komunikace s serverem, zatímco GUI zůstává reaktivní.
\end{itemize}

\section{Požadavky na překlad, spuštění a běh aplikace}
\subsection{Verze jazyků a nástrojů}

\subsubsection{Server - GO}

Aplikace pro Server byla napsána v jazyce GO, konkrétně ve verzi 1.21.

\subsubsection{Klient - Python}

Aplikeace pro Klient byla napsána v jazyce Python , konkrétně ve verzi 3.9.2. Pro klient, respektive jeho GUI byla použita knihovna Tkiner.

\subsection{Postup překladu}

\subsubsection{Server - GO}

Pro spuštění serveru je třeba mít nainstalovanou správnou verzi jazyka GO (1.21), otevřít adresář, kde se nechází hlavní soubor \texttt{server.go} a kompilovat program pomocí příkazu \texttt{make build} v příkazové řádce. Poté je nutno se přesunout do adresáře bin a server spustit \texttt{./server.go <IP\_adresa> <port>}

Program serveru byl napsán pro operační systém Linux.

\subsubsection{Klient - Python}

Pro spuštění klienta je třeba mít nainstalovanou správnou verzi jazyka Python (3.9.2), otevřít adresář, kde se nechází hlavní soubor \texttt{main.py} a spustit ho pomocí příkazu \texttt{python3 main.py} v příkazové řádce.

Program klienta byl napsán pro operační systém Linux a Windows.

Pokud je třeba doinstalovat knihovnu Tkiner. Je to možno provést pomocí příkazu \texttt{pip install Tk}.

\section{Závěr}
Semestrální práce byla vyvíjena v domácím prostředí a následně testována na školních zařízeních pro její správné odevzdání, tento proces trochu práci zkomplikoval, ale byla mi tak předána zkušenost s vývojem aplikace pro různá zařízení. Výsledná aplikace je funčkní jak na zařízeních v domácí síti, tak na zařízeních školních. Je možno zahrát se ve více lidech karetní hru blackjack s opakováním a s vlastnostmi krátkodobého a dlouhodobého odpojení.

\end{document}
